% \VignetteIndexEntry{AlgoTrader: an R package for working with
% trading algorithms} 
% \VignetteDepends{}
% \VignetteKeywords{finance, trading, algorithm}
% \VignettePackage{AlgoTrader}

% setwd('H:/user/R/Adrian/algotrader/inst/doc/')
% Sweave('AlgoTrader.Rnw');system("mypdflatex AlgoTrader")

%\documentclass{article}
\documentclass[nofootinbib]{revtex4}

\usepackage{Sweave}
\usepackage{hyperref}
\hypersetup{colorlinks,%
            citecolor=black,%
            linkcolor=blue,%
            urlcolor=blue,%
            }
\newcommand{\Rpackage}[1]{\textit{#1}}
\newcommand{\Rfunction}[1]{\textit{#1}}
\newcommand{\Robject}[1]{\texttt{#1}}
\newcommand{\Rclass}[1]{{\textit{#1}}}
\newcommand{\Rmethod}[1]{{\textit{#1}}}
\newcommand{\Rfunarg}[1]{{\textit{#1}}}
\newcommand{\R}{{\normalfont\textsf{R }}{}}
\newcommand{\classdef}[1]{%
  {\em #1}
}
\newcommand{\code}[1]{{\texttt{#1}}}
\newcommand{\file}[1]{{\texttt{#1}}}

\begin{document}

\title{AlgoTrader: an R package for trading algorithms}
\author{Adrian A. Dr\u{a}gulescu}
\email{adrian.dragulescu@gmail.com}
\date{\today, last modified on 1/5/2009}

\begin{abstract}
The \code{AlgoTrader} package provides a framework \R
offers a great environment for rapid trade
idea development and testing. 
\end{abstract}
%\pacs{R, algorithmic trading, prototype based programming}

\maketitle
\tableofcontents


%%%%%%%%%%%%%%%%%%%%%%%%%%%%%%%%%%%%%%%%%%%%%%%%%%%%%%%%%%%%%%%%%%%%%%%%%%
\section{Introduction}
%% \textcolor{red}{ texttext}

Financial trading is a complex activity that can be decomposed into
several distinct components: data acquisition, data analysis, model
building, decision making, risk analysis, order placement, execution,
and reporting.  Traditionally, a brokerage house provides the later
part of this trading flow while the steps leading to the decision
making are left to the trader himself.  For algorithmic trading 
computers are used to automate all the steps of the trading flow.

There are a lot of useful open-source \R packages that focus on
different aspects of trading.  Without trying to be comprehensive,
here are a few relevant packages (not all are on CRAN).
\begin{quote}
\begin{description}
\item[Data aquisition:]quantmod, IBrokers, RIB, RBloomberg, opentick
\item[Data analysis:]quantmod, TTR, xts, zoo, Rmetrics, tseries, etc. 
\item[Model building:] 
\item[Decision making:]
\item[Risk analysis:] PerformanceAnalytics
\item[Order placement:] IBrokers, RIB
\item[Reporting:] PerformanceAnalytics
\end{description}
\end{quote}

The lack of solutions for the \textbf{Model building} and
\textbf{Decision making} steps reflects their idiosyncratic nature.  A
user can combine any \R packages with his own code to arrive at a 
personal trading strategy.  It would be useful to have a consistent
\R framework to catalog, evaluate, and use trading ideas.  For
example, it should be easy to add a new trading rule to an existing
algorithm, and see it's effect.  Also, it should be easy to modify the
way every instrument position change under a given trade signal. 

The goal of the package {\tt\textbf{AlgoTrader}} is to provide such a
framework using an object oriented approach\footnote{See Section
  \ref{appendixProto} for a discussion on the design decision.}.  There
are two main classes.  An {\tt Algo} class from which all user
algorithms inherit and a {\tt Trader} class used for performace
tracking and reporting.  By convention, the parent classes {\tt Algo}
and {\tt Trader} are capitalized, while lower case names, say {\tt
  algo1} and {\tt trader1} are used for instances of these classes.
Conceptually, a model is represented by an algorithm and a trader.  

The functionality provided in this package allows the user to define
multiple trading algorithms.  He could gradually refine these
algorithms by using the inheritance mechanism, or use the same
algorithms with different parameters to instantiate different
\code{Algo} objects.

The \code{Algo} object contains all the historical data necessary, a
set of parameters, functions that generate trading signals, and a set
of functions that modify the positions given a trading signal.  Using
an algorithm {\tt algo}, the user can create a {\tt trader} that uses
the algoritm over a specified time interval to arrive at a trading
decision for each time step.  A trade log is kept by the trader, so
that an analysis of the trading performance can be done.  The
\code{trader} has different methods to summarize/plot/evaluate his
performance.  For strategy selection or for tuning the algorithm
parameters, the performance of several traders can be compared side by
side.  As the \code{AlgoTrader} package does backtesting, it is best
if only deterministic trading algorithms are used so performance
results are reproducible.

Once a trading strategy is in place, one can use it with a real-time
data feed, so that the trading decision made during the last time step
be sent to a broker for order placement and execution.  A word of
caution may be in order.  Depending on how the data feeds are set up
and the complexity of the trading algorithm, the current
implementation will likely not support a higher than 10 seconds
trading frequency.  



In Sections \ref{Algo} and \ref{Trader} the objects \code{Algo} and
\code{Trader} are discussed in detail.  Section \ref{appendixTF}
presents a simple trend following algorithm for one instrument, while
Section \ref{appendixLG} shows an algorithm with two instruments, a
stock and a call option.  If you are not familiar with the package
\code{proto} \cite{proto}, before diving into the next sections please
take the time to read the excellent vignettes and documentation
available for this package; at a minimum read over Section
\ref{appendixProto}.

%%%%%%%%%%%%%%%%%%%%%%%%%%%%%%%%%%%%%%%%%%%%%%%%%%%%%%%%%%%%%%%%%%%%%%%%%%
\section{Class {\tt Algo}}
\label{Algo}

The \code{Algo} object is a top level proto object which all user
algorithms implement.  It is conceptually equivalent to an abstract
class in the Java programming language.  By itself, an \code{Algo}
object is not useful because it is just an empty shell that the user
algo needs to fill.  It does not have a constructor method.

To create a new algorithm, one can start filling the required slots 
sequentially
\begin{Schunk}
\begin{Sinput}
> algo1 <- Algo$proto()              # announce that you inherit from Algo
> algo1$algoName <- "my best one!"   # you should put something more descriptive
> algo1$validSignalNames <- c("NONE", "BUY", "SELL", "LIQUIDATE")
> algo1$ls()
\end{Sinput}
\begin{Soutput}
[1] "algoName"         "validSignalNames"
\end{Soutput}
\end{Schunk}

It is required to declare what signal names your algorithm generates.
\code{"NONE"} is a reserved signal name that will do nothing,
essentially advancing the current state one time step into the future.
So unless you will have a trading decision at each time step, you will
need to specify \code{"NONE"} among your \code{validSignalNames}.

Signal functions are a collection of functions that get called at each
time step to determine the trading decision.  To avoid ambiguities,
there is only one posible trading signal at each time step.  You can
register as many signal functions as you need, the only requirement
being that they return one of the \code{validSignalNames} already
registered. It is best if the signal functions are defined in the same
\code{algo1} environment.  So assuming we have these functions
defined, we can register them 

\begin{Schunk}
\begin{Sinput}
> algo1$signalFunctions <- c(".buyFun", ".sellFun", ".liquidateFun")
\end{Sinput}
\end{Schunk}

For flexibility, it is not necessary to have a one to one
correspondence between a signal name and a signal function.  One
signal function can return several signal names at different time
steps, for example the same function can return a \code{"BUY"} or a
\code{"SELL"} signal at different time steps.

If one wants to implement several trading rules, each trading rule can
have its own signal function.  All signal functions are evaluated at
each time step.  As each signal function returns a signal name,
potentially there can be conflicting signals in the same time step.
This ambiguity is resolved using signal priorities.  The higher the
signal priority for a signal function is, the higher its precedence is
when the trading signal for that time step is decided.  Signal
priorities should be a vector of positive numbers and should have the
same length as the \code{signalFunctions}.  For example,  

\begin{Schunk}
\begin{Sinput}
> algo1$signalPriority <- c(1, 1, 2)
\end{Sinput}
\end{Schunk}

In this case, we set the signal function \code{".buyFun"} to have the
same priority as \code{".sellFun"}, but they are both trumped by the
signal function \code{".liquidateFun"} which has a higher priority.

The signal functions are expected to take an object of class
\code{Trader} as their first argument after the \code{.} argument that
denotes their own proto object.  This allows the implementation of
state dependent algorithms\footnote{By state dependent, I refer to
  algorithms that generate signals based on the current state of the
  portfolio/account.}.  For example, the algo can initiate a
\code{"LIQUIDATE"} signal if the current PnL is below a treshold, or
if there exists a cash/margin constraint.  One can easily think of
other such useful criteria.

\begin{table}
  \begin{center}
    \begin{tabular}{lll}
       Type & Name & Description \\ \hline
       variable & \code{algoName}  &  name of the algorithm, character \\
       variable & \code{validSignalNames} &  character vector \\
       variable & \code{signalFunctions}  &  names of signal functions, 
         vector of characters \\
       variable & \code{signalPriority}   &  numeric vector, same
         length as \code{signalFunctions} \\ 
       variable & \code{quantityFunctions} & character vector, named
         with \code{validSignalNames} \\
       method   & \code{getPrices} & return instrument prices in
         current period as a named vector \\
       method   & \code{validate} & performs validity checks on the
         algo object.  \\
       \hline
    \end{tabular}
  \end{center}
  \caption{A description of required fields for each \code{Algo} object.}
  \label{tab:Algo}
\end{table}  


%%%%%%%%%%%%%%%%%%%%%%%%%%%%%%%%%%%%%%%%%%%%%%%%%%%%%%%%%%%%%%%%%%%%%%%%%%
%%%%%%%%%%%%%%%%%%%%%%%%%%%%%%%%%%%%%%%%%%%%%%%%%%%%%%%%%%%%%%%%%%%%%%%%%%
\section{Class {\tt Trader}}
\label{Trader}

As mentioned in the introduction, analysis of a trading algorithm
consists from an user defined \code{algo} object together with a user
defined \code{trader} object.  If the \code{Algo} object has all the
logic of the algorithm, the proto object \code{Trader} holds all the
methods that allow to report on the algorithm performace for a
collection of time steps.

Internally, when the method \code{trade} is called, two nested for
loops are executed.  The first, outer loop is over the time steps
while the second, inner loop is over all the signal functions of the
algorithm.  Looping in R usually creates a performance bottleneck, and
having state-dependent algorithms requires that the trader object is
passed to each signal function.  As \R does not support partial
function application (currying), the only way to avoid passing large
\code{Trader} objects back and forth to each signal function is to use
references.

In fact, because the signal functions can be ordered based on their
priority, not all the functions are needed to be evaluated at each
time step.  You evaluate until you get one function that does not
return "NONE".  In an algo with many trading rules, this can shorten
the run time significantly.


\begin{table}
  \begin{center}
    \begin{tabular}{lll}
       Type & Name & Description \\ \hline
       variable & \code{currentBar}  &  current time step, class POSIXct \\
       variable & \code{barIndex} &  a POSIXct vector of time steps  \\
       variable & \code{tradingSignal}  &  character vector with
       trading signal, same length as \code{barIndex} \\
       variable & \code{PnL}   &  list with performance details, 
         length as \code{barIndex} \\ 
       variable & \code{...} & other variables, see documentation \\
       method   & \code{new} & constructor for the class \\
       method   & \code{trade} & get trading signals, calculate performance \\
       method   & \code{summary} & summarize the object Trader \\
       method   & \code{plot} & attempts to plot the trading signals, prices, 
         quantities \\
       method & \code{extend} & extend a trader with another set of
         time steps \\
       method & \code{getPnL} & returns the profit and loss as an xts object
         \\
       method & \code{getPositions} & returns the positions as an xts object
         \\
         
       \hline
    \end{tabular}
  \end{center}
  \caption{A partial description of the \code{Trader} object and some
    of its methods.}
  \label{tab:Algo}
\end{table}  

Variables: currentBar, barIndex, tradingSignal, PnL, prevQty
Methods: trade, plot, summary, print.
Getter methods: getPnL, getPositions. 

MTM can be calculated easily from 


%%%%%%%%%%%%%%%%%%%%%%%%%%%%%%%%%%%%%%%%%%%%%%%%%%%%%%%%%%%%%%%%%%%%%%%%%%
%%%%%%%%%%%%%%%%%%%%%%%%%%%%%%%%%%%%%%%%%%%%%%%%%%%%%%%%%%%%%%%%%%%%%%%%%%
\section{Examples}

%%%%%%%%%%%%%%%%%%%%%%%%%%%%%%%%%%%%%%%%%%%%%%%%%%%%%%%%%%%%%%%%%%%%%%%%%%
\subsection{A trend following algo}
\label{appendixTF}

As an example of implementing a trading algo, I will present a very
simple trend following algorithm.  It has three available signals
\code{c("NONE", "BUY", "SELL")}.  A buy signal is generated if the
last price of the instrument has been above a moving average for a
given number of time periods.  Similarly, a sell signal is generated
if the last price of the instrument has been below a moving average
for a given number of time periods.


%%%%%%%%%%%%%%%%%%%%%%%%%%%%%%%%%%%%%%%%%%%%%%%%%%%%%%%%%%%%%%%%%%%%%%%%%%
\subsection{A long gamma algo}
\label{appendixLG}

This section presents a simple trading strategy that uses two
instruments and one signal function.  We consider a strategy with a
short position in a stock and a long position in a call option.  The
portfolio is overall long gamma.  We open the trade by being delta
neutral and as the stock moves and the option becomes more in or out
of the money we rebalance (hedge) the portfolio by buying or selling
the stock.  We decide to trade only the stock as it has a smaller
bid/ask spread and is more liquid.

Such a strategy works when the the implied volatility of the option
ends up being higher than the realized volatility.  A scenario that
will make money is if the stock has a large one directional move, or
if it crosses several times the strike of the option before the option
expires.  As we initiate the trade by being long the call options, we
need to recoup the initial capital overlay required to purchase the
options.

We choose to rebalance the portfolio only when the net delta of the
portfolio is a fraction of the initial stock delta.  The algorithm has
only two valid signals \code{c("NONE", "REBALANCE")}, and only one
signal function that determines when it's time to rebalance.  Because
there is only one signal other then \code{"NONE"}, we have only one
function that determines the positions after the \code{"REBALANCE"}
signal.  We need to implement the \code{getPrices} method to get the
prices of the instruments in the current period.  

Here is the code that implements this simple trading strategy.  The
class name is \code{LG1} (long gamma), and inherits from the abstract
class \code{Algo}.  See also the source file \code{example\_LG1.R} and
\code{?example\_LG1}.

\begin{Schunk}
\begin{Soutput}
LG1 <- Algo$proto()  # inherit from Algo

LG1$algoName         = "Long Gamma (long call, short stock) 1"
LG1$validSignalNames = c("NONE", "REBALANCE")

# Required getter to extract the prices for a given barIndex
LG1$getPrices <- function(., currentBar){
  structure(as.numeric(.$hdata[currentBar,c("Close.stock", "Close.call")]),
            names=c("stock", "call"))
}

# Register the signal functions.
LG1$signalFunctions  = c(".rebalanceFun")
LG1$signalPriority   = c(1)      # priority 0 is for signal "NONE"

# and the mapping between signal name and the quantity function. 
LG1$quantityFunctions <- structure(
   c(".qtyRebalanceFun"), names = c("REBALANCE")    
)

# The quantity for this bar for a trading signal.
LG1$.qtyRebalanceFun <- function(., trader)
{
  netQty <- trader$prevQty["stock"] +
    trader$prevQty["call"]*.$hdata[trader$currentBar, "delta"]

  Qty <- trader$prevQty
  Qty["stock"] <- Qty["stock"] - netQty

  Qty
}

# the REBALANCE signal function
LG1$.rebalanceFun <- function(., trader)
{
  signal <- "NONE"

  q0Stock <- abs(trader$q0["stock"])      # q0Stock is negative
  netQty  <- trader$prevQty["stock"] +
    trader$prevQty["call"]*.$hdata[trader$currentBar, "delta"]

  if ((netQty >= .$FRACTION*q0Stock) | (netQty <= -.$FRACTION*q0Stock))
    signal <- "REBALANCE"    
    
  return( signal )
}

# constructor method            
LG1$new <- function(., x, FRACTION = 0.20)
{  
  .$proto(
    FRACTION = FRACTION,     # when to rebalance
    hdata = x
  )
}
\end{Soutput}
\end{Schunk}








%%%%%%%%%%%%%%%%%%%%%%%%%%%%%%%%%%%%%%%%%%%%%%%%%%%%%%%%%%%%%%%%%%%%%%%%%%
%%%%%%%%%%%%%%%%%%%%%%%%%%%%%%%%%%%%%%%%%%%%%%%%%%%%%%%%%%%%%%%%%%%%%%%%%%
\section{Conclusions}
\label{Conclusions}

We provide a prototype based framework for defining, evaluating and
using trading algorithms.  As mentioned in \cite{KatesPetzoldt}
prototype programing is particularly suitable for situations with
small number of objects and with frequent exceptions.  This is exactly
the the case when working with trading algorithms.

The implementation is dependent on the proto package, so the user has
to learn about proto objects and proto programming, but this can be
overcome quickly and benefits are worthwile.  The user has the ability
to overwrite or customize all the available methods that ship with the
package.   For example, the \code{Trader\$plot} method is one such
candidate for user customization.

Can the proposed \code{AlgoTrader} framework deal with adaptative
algorithms?  That is, can the algo automatically suggest trading a
different instrument, or it is limited the same set of instruments?
There is no such limitation as long as the algo methods can gracefully
exit one instrument and start trading a different one.  For example,
algorithms that involve instruments with a finite life (e.g. options,
futures, etc.) need to explicitly create a switch signal, define the
quantity functions, etc.  

The present framework can be extended in several ways.  It may be of
interest to calculate the margin requirements of the existing position
at each time step, so \code{Trader} class could have a method to
calculate the so called "Reg T" margin.  Also, a risk metric like VaR
can be useful to report.  Then, users can define algorithms that
trigger decision based on margin or VaR.  Having methods to compare
two traders would also be interesting.  Further analyzing the returns
of the trader with the \code{Performance Analytics}\cite{PA} package
can also provide useful insights.

Interested parties are encouraged to contribute code and suggest
improvements to the existing \code{AlgoTrader} package, in order to
advance the use of open source \R in finance.



%%%%%%%%%%%%%%%%%%%%%%%%%%%%%%%%%%%%%%%%%%%%%%%%%%%%%%%%%%%%%%%%%%%%%%%%%%
%%%%%%%%%%%%%%%%%%%%%%%%%%%%%%%%%%%%%%%%%%%%%%%%%%%%%%%%%%%%%%%%%%%%%%%%%%
\section{Appendix: proto objects}
\label{appendixProto}

I experimented with different approaches for the internals of the {\tt
  AlgoTrader} package.  An initial implementation used {\tt S4}
classes and methods, but I was unsatisfied with performance and the
resulting coding style.  The performance bottlenecks came from {\tt
  R}'s copy-on-modify paradigm and the lack of pass by reference
support.  While there are some advantages of using {\tt S4} classes,
in the end I felt unsatisfied and looked for other ways of doing what
I wanted.

Other object-oriented choices available in {\tt R} are the {\tt
  R.oo}\cite{Roo} package and the {\tt proto}\cite{proto} package.  In
the end I chose the {\tt proto} package after reading the excellent
documentation available for this package that clarified some subtle
points.

To show how easy and natural it is to work with proto objects, I
present the definition of a {\tt Circle} class.
\begin{Schunk}
\begin{Sinput}
> Circle <- proto(
   noCircles = 0,                      # class variable
   radius    = numeric(),              # instance variable
   area = function(.){pi*.$radius^2},  # class method
   new  = function(., radius){         # define a constructor
     .$noCircles <- .$noCircles + 1    
     .$proto(radius = radius)          # return the object
   })
\end{Sinput}
\end{Schunk}
The \code{.} refers to the current object and is equivalent to the
Java's \code{this} keyword.  You can create two circle objects by
calling the {\tt new} method.
\begin{Schunk}
\begin{Sinput}
> c1 <- Circle$new(1)
> c2 <- Circle$new(2)
\end{Sinput}
\end{Schunk}
And now explore the objects
\begin{Schunk}
\begin{Sinput}
> c1$radius          # = 1
> c1$area()          # = pi
> Circle$noCircles   # = 2, two circles created
\end{Sinput}
\end{Schunk}
%$

Because proto objects are references stored internally as
environments, debugging or updating a method is very convenient.  For
example, suppose you want to modify your signal functions.  You just
evaluate the modified algo method, and all the objects that point to
it will use the updated algo method.

As prototype programming is not really main stream, but has common
features with object oriented programming which is much better know, I
will tend to use language that matches an object oriented approach.
While this can be slightly inaccurate, it should help the reading of
the paper.



%%%%%%%%%%%%%%%%%%%%%%%%%%%%%%%%%%%%%%%%%%%%%%%%%%%%%%%%%%%%%%%%%%%%%%%%%%
%%%%%%%%%%%%%%%%%%%%%%%%%%%%%%%%%%%%%%%%%%%%%%%%%%%%%%%%%%%%%%%%%%%%%%%%%%
\begin{thebibliography}{99}

\bibitem{R} R Development Core Team:
\emph{R: A Language and Environment for Statistical Computing},
R Foundation for Statistical Computing, Vienna, Austria.
ISBN 3-900051-07-0, URL \url{http://www.R-project.org}

\bibitem{IBrokers}Jeffrey A. Ryan: \emph{IBrokers: R API to
  Interactive Brokers Trader Workstation}, R package version 0.1-0,
  2008

\bibitem{proto}Louis Kates and Thomas Petzoldt (2007) \emph{proto:
  Prototype object-based programming.} R package version
  0.3-8. URL \url{http://code.google.com/p/r-proto/}
  
\bibitem{KatesPetzoldt} Kates, L., and Petzoldt, T. (2004)
  "Prototype-Based Programming in Statistical Computation.", URL
  \url{http://r-proto.googlecode.com/files/prototype_approaches.pdf}.

\bibitem{Roo} Bengtsson, H., (2003) \emph{The R.oo package - Object-Oriented
  Programming with References Using Standard R Code}, Proceedings of
  the 3rd International Workshop on Distributed Statistical Computing
  (DSC 2003), ISSN 1609-395X, Hornik, K.; Leisch, F. \& Zeileis,
  A. (ed.)

\bibitem{PA} Peter Carl and Brian G. Peterson
  (2008). PerformanceAnalytics: Econometric tools for performance and
  risk analysis.. R package version 0.9.7.1.
  \url{http://braverock.com/R/}
  
\end{thebibliography}

\end{document}
