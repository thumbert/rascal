% \documentclass[preprint, 14pt]{revtex4}
\documentclass[12pt]{article}
\usepackage[margin=1in]{geometry}

\setlength{\parindent}{0pt}

\begin{document}
\large
\title{Words to a departing friend}
\author{AAD}
\date{February 1, 2024}
\maketitle

Hi everyone,

\bigskip
For those of you who don't know me, my name is Adrian and I'm going 
to say a few words about my friend Alin.  


\bigskip
I've met Alin for the first time when I landed at Dulles Airport on 
16 Aug 1996.  I was 22 and came here to do a PhD in Physics.  He was 
waiting for me with a handmade sign because, you know, in those days 
the internet was just getting popular and nobody was thinking it 
was a good idea to send pictures of yourself over the internet.  
He was already a PhD student in the Economics 
department at the University of Maryland, arriving in College Park one 
year earlier.  Monica came a few months before me and by some lucky 
coincidence, my mother was her PhD advisor in Romania.  That's how I 
ended up in their second bedroom for a couple a weeks until I moved 
to my own place.  It was Alin who recommended me his former roommate, 
a friendly japanese named Tsune with whom I shared an apartment for 
the next 5 years. 

\bigskip
For those of us here in the 50+ age group, looking back at ourselves 
as we were in our early twenties can be dizzying, even reckless if 
we're not prepared for the follow-up questions, for the reckoning.  
Are we comfortable with where we are now?  With what we've done with 
our life, with our family, with our kids?  How we carried ourselves 
in challenging moments?  I am convinced that Alin would have no 
second-thoughts about any of these questions.  Not that it always 
had been easy, or certain.  But I am sure he would have done it 
all over again exactly the same way, because {\bf why not}, what 
could have turned out materially better?  

\bigskip
Alin carried himself everywhere with a tremendous ease, with an 
unstudied legerity, displaying a palpable virtue.  But I'll 
spare you the pompous language.  You know that I've never seen him 
lose his temper?  For example, I don't even remember 
him honking the car horn.  We're Romanians, we {\bf don't get} our 
driver's license unless we show that we can proficiently honk!  
OK, OK, I've seen him getting frustrated.  Many times.  It happened 
when he was faced with crass incompetence, especially if it was 
broadcasted from the desk of a self-appointed expert.  But even 
then, he would shake his head and say Tataie, and patiently explain 
to you what you were missing.    

\bigskip
He was passionate about all sort of things.  Things that his intellect 
would get hooked on.  And then he would spend hours researching them 
up late into the night, because, you know ... {\bf someone is wrong 
on the internet}. 

\bigskip
Alin loved to travel.  But he wasn't sold on airplanes. What, 
you ask?  For most of us, commercial aviation was a solved problem 
sometime in the 60's.  He probably read something on the internet 
(in one of those nights when he stayed up late) that convinced him that 
air flight was not for him or his family.  {\bf But} when going to Europe
it had to be by air, so maybe Delta or a specific type of aircraft would 
cut it.  Have you listen to the news recently?  Look what's happening 
to Boeing now, at all the emergency landings!  So who was right all long, 
huh?  So then {\bf driving} it was when it was time for traveling...  
This guy, he criss-crossed the US {\bf and} Europe by car on several, 
yes, you heard it, {\bf several} grand tours.  

\bigskip
He had a real talent and passion for financial investing.  I was also 
interested in that, so we both opened electronic trading accounts in '98 
(a novelty at the time) to do day trading.  I remember that our first 
electronic brokerage firm was called Suretrade.  Let me be clear.  There 
was nothing sure about it.  You would place an order and not get a 
confirmation until minutes later.  God forbid you placed a cancel order.  
You would routinely get disconnected when there was too much traffic.  
A nightmare.  Unbeknown to us at the time, we were riding out the dot-com boom.  
I didn't have much money.  My annual stipend as a Research Assistant 
was \$15,000 of which I was paying about \$600/month for housing, 
so not a lot was left.  Several days a week we would go to a computer lab 
at UMD, read the news before the market opened and trade until noon or so.  
We lurked in dubious chat rooms and followed tips from a fellow with 
the screen name: Cheetah.  I know, it was not a fair and acceptable 
use of the computing infrastructure of the University of Maryland.  
And also, we didn't make a lot of money.  But what we surely did in those 
years was to fill out a really long Schedule D on our tax returns in 
April to record our capital gains and losses.

\bigskip
And what better proof that Alin was great at decision making and long term 
investing than the fact that he built his life and family with Monica.  
She was the binding agent that made everything better: loving, 
caring, devoted.  She never lost hope of things turning for the better 
and did everything possible to help Alin to his very last breath.  Monica, 
heavy grief will be in your heart for a long time, so take care, heal; 
we love you.  

\bigskip
Oh, did you know that Alin's PhD took about 9 or 10 years to complete?  
Talk about perseverance!  He had 3 advisors (I only had one!)  One advisor 
had to move out from the country, one passed-away, and finally the third one 
was the right one.  They told him he held an unofficial record in the 
department with his 10 years 3 advisors doctorate.    

\bigskip
Now, I want to address all the Romanian daughters and sons, and in 
particular, Oana and Alina.  I know, I know, we Romanian parents, 
we don't get the respect we should around here.  Everybody heard of 
Chinese 'Tiger Moms'.  But hear me out, I know what I'm talking about.   
I'm a recovering Romanian parent. Alin was the best of us.  
He argued with teachers about homework, he criticized the curriculum 
like it was made by some noobs without any experience or talent.  
Hear me: We want so much for you because we had too 
little. In this land of plenty, it's hard to stay hungry.  
And if you think we're on your case, and tell you that time is short 
and you need to stay focused and learn, and absorb, and grow, 
it's because we {\bf care} and we're right!  Alin would agree with me 
on that.  He was so proud of you Oana, of your special powers, and of 
you Alina, his secret and unfinished project, telling me he needs to 
hold you back a bit because you go at it too hard.  But I'm telling 
you here, Alina, don't stop, go at it as hard as you can, give 
it a good run! 


\bigskip
While I was in school Alin and Monica's house was my surrogate home.  
In my first 5 or 6 years in the US, all major holidays were spent 
together with them.  We had an active Romanian student group at 
UMD and we liked to do things together, parties, events, picnics.  
We would see each other very often.  As I was wrapping up 
my dissertation, Alin hired me out of school to 
work part time as a contractor in his group at the IMF.  So I owe him 
my first job.  He's also the one who advised me in my negotiations 
for my next job while insisting to have a full-time counter offer 
from the IMF.  That helped me get a better starting salary and perks in 
my job with Constellation Energy where I've been since 2002.  
Alin and Monica are the godparents of my two girls.  We've been on 
many vacations together.  But as our girls grew up, and their schedules 
took precedence (remember the previous explanatory section regarding 
Romanian parenting?), we neglected so make time for us together.  
We saw each other less, only a couple of times a year.  We still 
had plenty of long and close phone calls, but life was moving ahead. 

\bigskip
It is bittersweet to see here today faces of friends I haven't 
seen in so so long.  I know Alin would have loved it too, to see 
all of you, together, and again.  It's wonderful to read the lines on 
the Tribute Wall, from old time friends, from Romania, from highschool, 
from work.  To read that Alin was {\bf your} best friend.  Well, 
wait a minute, I say! I thought {\bf I} was Alin's best friend.  
The math just doesn't work out.  But I'm not surprised to see he made 
so many other people feel this special. 

\bigskip
Fate arranged for me to see Alin one last time this past August, in 
Hilton Head South Carolina, on a Thursday evening, while out for dinner, 
in a restaurant I picked on the spur of the moment that afternoon. 
Our reservations were 15-minutes apart, tables right across.  I had 
to look more than twice to convince myself that I see him there, next to 
Monica and his two girls, happy and chatty as usual.  It was a gift 
born of randomness that I will treasure forever. 

\bigskip
Alin, our daughters will graduate highschool next year.  Our schedules 
will suddenly open.  You surely must have forgotten we had plans!  
And it's on me, {\bf I} should have reminded you!  Maybe then this 
wouldn't have happened.  You wouldn't have gone so soon.  Maybe, 
maybe we would have started day trading again!  Suretrade is no more, 
but what {\bf is} sure is that I'll miss you dear friend and that I'll 
think of you and wish you were still around so we won't feel so 
utterly empty. 

\bigskip
Thank you


\end{document}