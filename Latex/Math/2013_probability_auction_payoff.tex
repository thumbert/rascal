\documentclass[12pt]{article}

\begin{document}
\title{Auctions with random payoffs}
\author{Adrian A. Dr\u{a}gulescu}
\date{September 30, 2013}
\maketitle



\section{Distribution of the payoff}
Consider an auction where a participant is allowed to bid price,
quantiy pairs $(b_i, q_i)$ with $i=1\to N$.  We'll assume that the bid
prices $b_i$ are ordered decreasingly. 

The auction clears at a random clearing price $c$, and the participant
gets an award of size $\sum_i q_i$ for all bids $i$ such that $b_i
\geq c$.  We assume that in the case when the clearing price $c$
exactly equals one of the bids $b_i$, the entire quantity $q_i$ is
awarded.

The auction payout for each unit quantity is a random variable $S$
which represents the settlement price.  The values of $S$ denoted $s$
can be both positive or negative.  The random variable $C$
corresponding to the clearing price has a probability density ${\cal
  P}_C(c)$ and the random variable $S$ corresponding to the settlement
price has a probability density ${\cal P}_S(s)$ considered independent
of each other.  As in the case of $S$, the clearing prices $c$ can be
both positive or negative. 

The payout of the auction can then be written as
\begin{equation}
  \pi = (s - c)\sum_{i=1}^N q_i (1 - \theta(c - b_i)) 
\end{equation}
where $\theta$ is the Heaviside step-function.  

The problem is to find the probability density of the auction payout
$\Pi$.  This probability density can be written as
\begin{equation}
  {\cal P}(\pi) = \int_{-\infty}^{\infty}ds \int_{-\infty}^{\infty}dc\,
    {\cal P}(s)\,{\cal P}(c)\, 
    \delta(\pi - (s - c)\sum_{i=1}^N q_i (1 - \theta(c-b_i))) 
\end{equation}
where $\delta$ is the Dirac distribution function. 

Let's focus first on the case when $N=1$.  You can split the integral
over $c$ into two parts from $(-\infty,b)$ and from $(b,\infty)$ to
get
\begin{eqnarray*}
  \lefteqn{{\cal P}(\pi) = \int_{-\infty}^{\infty}ds \int_{-\infty}^{b}dc
    {\cal P}(s) {\cal P}(c) \delta(\pi - q(s - c)) }\\ 
  && {} + \int_{-\infty}^{\infty}ds \int_{b}^{\infty}dc
    {\cal P}(s) {\cal P}(c) \delta(\pi)
\end{eqnarray*}
The first integral over $s$ in the first term can be taken by
resolving the $\delta$ function to get
\begin{equation}
  {\cal P}(\pi) = \frac{1}{q}\int_{-\infty}^{b}dc
    {\cal P}_S(c+\pi/q) {\cal P}_C(c) + P_0\delta(\pi) 
\end{equation}

where $$P_0 = \int_{b}^{\infty}dc\,{\cal P}_C(c) $$ is a constant that
represents the probability of not clearing anything in the auction,
in the case when the clearing price $c$ is greater than the bid
price $b$.




What is interesting is that the average payout can be found out to be 
\begin{equation}
  \langle \Pi \rangle = \int_{-\infty}^{\infty}d\pi\,\pi\,{\cal P}(\pi) 
     = q\int_{-\infty}^{b}dc\,{\cal P}_C(c)\,(\langle S \rangle - c) 
\end{equation}


Similar arguments can be made for the case of multiple bids.  For
example, for two bids we get
\begin{eqnarray*}
  \lefteqn{{\cal P}(\pi) = 
    \frac{1}{q_1+q_2}\int_{-\infty}^{b_2}dc\,{\cal P}_S(c+\pi/(q_1+q_2))\,
       {\cal P}_C(c)} \\ 
  && {} + \frac{1}{q_1}\int_{b_2}^{b_1}dc\,{\cal P}_S(c+\pi/q_1)\, {\cal P}_C(c) \\
  && {} + P_0\delta(\pi) 
\end{eqnarray*}

\section{Portfolio considerations}
It is interesting to consider portfolios of tickets corresponding to
different auctions. 



\end{document}
